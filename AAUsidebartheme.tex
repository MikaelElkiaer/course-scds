\documentclass[10pt]{beamer}
\usetheme[
%%% options passed to the outer theme
%    hidetitle,           % hide the (short) title in the sidebar
%    hideauthor,          % hide the (short) author in the sidebar
%    hideinstitute,       % hide the (short) institute in the bottom of the sidebar
%    shownavsym,          % show the navigation symbols
%    width=2cm,           % width of the sidebar (default is 2 cm)
%    hideothersubsections,% hide all subsections but the subsections in the current section
%    hideallsubsections,  % hide all subsections
    left               % right of left position of sidebar (default is right)
%%% options passed to the color theme
%    lightheaderbg,       % use a light header background
  ]{AAUsidebar}

% If you want to change the colors of the various elements in the theme, edit and uncomment the following lines
% Change the bar and sidebar colors:
%\setbeamercolor{AAUsidebar}{fg=red!20,bg=red}
%\setbeamercolor{sidebar}{bg=red!20}
% Change the color of the structural elements:
%\setbeamercolor{structure}{fg=red}
% Change the frame title text color:
%\setbeamercolor{frametitle}{fg=blue}
% Change the normal text color background:
%\setbeamercolor{normal text}{bg=gray!10}
% ... and you can of course change a lot more - see the beamer user manual.


\usepackage[utf8]{inputenc}
\usepackage[english]{babel}
\usepackage[T1]{fontenc}
% Or whatever. Note that the encoding and the font should match. If T1
% does not look nice, try deleting the line with the fontenc.
\usepackage{helvet}

% colored hyperlinks
\newcommand{\chref}[2]{%
  \href{#1}{{\usebeamercolor[bg]{AAUsidebar}#2}}%
}

\title[Decentralized Label Model]% optional, use only with long paper titles
{A Decentralized Model for Information Flow Control}

\subtitle{Andrew C. Myers and Barbara Liskov, 1997}  % could also be a conference name

\date{September 23, 2015}

\author[Mikael Elki\ae r Christensen] % optional, use only with lots of authors
{
  Mikael Elki\ae r Christensen\\
  \href{mailto:michri11@student.aau.dk}{{\tt michri11@student.aau.dk}}
}
% - Give the names in the same order as they appear in the paper.
% - Use the \inst{?} command only if the authors have different
%   affiliation. See the beamer manual for an example

\institute[
%  {\includegraphics[scale=0.2]{aau_segl}}\\ %insert a company, department or university logo
  Department of Computer Science\\
  Aalborg University\\
  Denmark
] % optional - is placed in the bottom of the sidebar on every slide
{% is placed on the title page
  Department of Computer Science\\
  Aalborg University\\
  Denmark
  
  %there must be an empty line above this line - otherwise some unwanted space is added between the university and the country (I do not know why;( )
}


% specify a logo on the titlepage (you can specify additional logos an include them in 
% institute command below
\pgfdeclareimage[height=1.5cm]{titlepagelogo}{AAUgraphics/aau_logo_new} % placed on the title page
%\pgfdeclareimage[height=1.5cm]{titlepagelogo2}{graphics/aau_logo_new} % placed on the title page
\titlegraphic{% is placed on the bottom of the title page
  \pgfuseimage{titlepagelogo}
%  \hspace{1cm}\pgfuseimage{titlepagelogo2}
}


\begin{document}
% the titlepage
{\aauwavesbg%
\begin{frame}[plain,noframenumbering] % the plain option removes the sidebar and header from the title page
  \titlepage
\end{frame}}
%%%%%%%%%%%%%%%%

%% TOC
%\begin{frame}{Agenda}{}
%\tableofcontents
%\end{frame}
%%%%%%%%%%%%%%%%

\section{Introduction}
% motivation for creating this theme
\begin{frame}{Introduction}{}
  The present beamer theme called the \alert{AAU Sidebar Beamer Theme} is an attempt to
  \begin{itemize}
    \item<1-> create a simple and elegant beamer theme which can be used by students and researchers affiliated with Aalborg University (AAU),
    \item<2-> create a unique AAU theme which does not resemble any of the standard beamer themes. People should associate this theme with AAU and not with beamer,
    \item<3-> keep the amount of clutter to a minimum. Only the important things should be on the slides,
    \item<4-> retain the powerful customisation tools provided by the template system of the beamer class.
  \end{itemize}
\end{frame}
%%%%%%%%%%%%%%%%

\subsection{License}
% the license
\begin{frame}{Introduction}{License}
  \begin{itemize}
    \item<1-> The AAU logo is covered by copyright rules. I have used the logo from \chref{http://aau.designguides.dk}{http://aau.designguides.dk}. As long as you use the theme for making presentations in connection with your work at AAU, you are allowed to use the AAU logo.
    \item<2-> The rest of the theme is provided under the GNU General Public License v. 3 (GPLv3). This basically means that you can redistribute it and/or modify it under the same license. For more information on the GPL license see \chref{http://www.gnu.org/licenses/}{http://www.gnu.org/licenses/}
  \end{itemize}
\end{frame}
%%%%%%%%%%%%%%%%

{\aauwavesbg
\begin{frame}[plain,noframenumbering]
  \finalpage{Questions?}
\end{frame}}
%%%%%%%%%%%%%%%%

\end{document}
